\gls{cppql} is a C++ library that wraps around the \gls{sqlite} C API. It offers multiple levels of abstraction.

At the lowest level, it merely provides two classes to handle the lifetimes of database connections and \marginlabel{See Section~\ref{section:database} and \ref{section:statements}.}\gls{statement}s. You call all the C functions yourself with the handles you \code{get} from the objects. Obviously, this is not very interesting.

At a higher level the various classes expose a number of methods corresponding to the most common functions from the C API. Additionally, \marginlabel{See Section~\ref{section:tables}.}creating tables is made a little bit easier through a more object-oriented approach. You'll still be writing most, if not all of your own SQL statements, though.

At the highest level a fully type safe, C++ only \marginlabel{See Section~\ref{section:typed_tables}.}interface is available. To select rows from a table, you write typical C++ expressions with objects representing columns, e.g. \code{table.select(col1 > 10)}, and iterate over the results. Similar functionality is available for deletions, inserts, etc. Note that advanced queries, such as joins, are not supported through this high level interface. For that, you must use the lower level functionality.

\gls{cppql} does not wrap every single C function. You might therefore find yourself having to use the internal handles to call more obscure C functions that are not supported by the C++ layers. It also does not do e.g. exhaustive type checking when constructing string based queries, instead returning \gls{sqlite} \marginlabel{See Section~\ref{section:errors}.}error codes.

For build instructions, see Section~\ref{section:build}.

\subsection{Acknowledgements}
\label{section:introduction:acknowledgements}

\gls{cppql} currently relies on the following libraries:
\begin{itemize}
	\item fmt 7.0.3\cite{fmt}.
	\item nlohmann/json\cite{nlohmann}.
	\item \gls{sqlite}\cite{sqlite}.
\end{itemize}