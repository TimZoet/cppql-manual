The \gls{table_class} class does not offer much in terms of type safety or extensive interfaces. For that, there is the \gls{typed_table_class} class. It can be constructed by passing it a normal table and allows you to specify a static type for each column. See Listing~\ref{lst:typed_tables:typed}.

\lstinputlisting[caption={Creating a typed table.}, label={lst:typed_tables:typed}]{snippets/create_typed_table.cpp}

The column types follow the same rules as column retrieval. See Section~\ref{section:statement:get}. If you pass an invalid type, an exception will be thrown directly upon construction.

% TODO: Conversions between differently sized ints, nulls, etc, during all operations (select, insert, update, etc.).
% Binding values: value (fixed) vs pointers (dynamic).
% Order by: -col1 + col2 becomes ORDER BY col1 DESC, col2 ASC;
% How do text and blobs work?

\subsection{Filter}

\subsection{Order By}

