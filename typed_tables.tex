The \gls{table_class} class does not offer much in terms of type safety or extensive interfaces. For that, there is the \gls{typed_table_class} class, which can be used to construct queries without actually writing any SQL code.

\subsection{Creating Typed Tables}
\label{section:typed_tables:create}

A \gls{typed_table_class} can be constructed by passing it a normal table and allows you to specify a static type for each column. See Listing~\ref{lst:typed_tables:create}.

\lstinputlisting[caption={Creating a typed table.}, label={lst:typed_tables:create}]{snippets/create_typed_table.cpp}

The column types follow the same rules as column retrieval. See Section~\ref{section:statement:get}. If you pass an invalid type, an exception will be thrown directly upon construction.

\subsection{Queries}
\label{section:typed_tables:queries}

The \code{INSERT}, \code{DELETE}, \code{SELECT} and \code{UPDATE} statements can all be constructed directly from a \gls{typed_table_class}. They all offer a C++ only interface to the underlying table. Each type of query is documented in a separate section.

% Conversions between differently sized ints, nulls, etc, during all operations (select, insert, update, etc.).
% Query objects only valid as long as table they operate on is valid.

\subsection{Filter}

% Binding values: value (fixed) vs pointers (dynamic).

\subsection{Order By}

% Order by: -col1 + col2 becomes ORDER BY col1 DESC, col2 ASC;