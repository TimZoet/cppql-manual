To remove rows from a table, you can construct a \gls{delete_class} object from a \gls{typed_table_class}. This objects holds a statement of the form \newline \code{DELETE FROM <table> WHERE <expr>}. See Listing~\ref{lst:delete:create}.

\lstinputlisting[caption={Creating a delete statement.}, label={lst:delete:create}]{snippets/delete_create.cpp}

The first parameter is the expression by which to \marginlabel{See Section~\ref{section:typed_tables:filter}.}filter. The second parameter indicates whether expression parameters should be bound immediately upon creation. After construction, the object can be invoked one or more times. See Listing~\ref{lst:delete:call}.

\lstinputlisting[caption={Deleting multiple rows.}, label={lst:delete:call}]{snippets/delete_call.cpp}

There are two overloads of \code{operator()}. The first does not take any parameters and runs the delete statement without (re)binding any parameters. The second accepts a boolean parameter which, if true, will result in all parameters being bound before running the delete. If the filter expression does not have dynamic parameters it is best to bind upon construction and then invoke the parameterless function. This will result in the best performance.

\subsection{Order By and Limit}
