cppql requires a compiler that supports C++20. It has been tested on:

\begin{itemize}
	\item Windows 10 with MSVC 19.28.
	\item Ubuntu 20.04 with GCC 11.
\end{itemize}

\subsection{Dependencies}

To build \gls{cppql} you must have several packages installed. On Windows, you could use \href{https://github.com/microsoft/vcpkg}{vcpkg} to install them. On Linux, \href{https://conan.io}{Conan}. Not all packages are required for all configurations. \code{cppql} is guaranteed to work with the following versions of these packages:

\begin{itemize}
	\item fmt 7.0.3\cite{fmt}. Always required. (To be replaced by C++20 fmt functions.)
	\item nlohmann/json 3.9.1\cite{nlohmann}. Only required when building tests.
	\item sqlite 3.35.0\cite{sqlite}. Always required.
\end{itemize}

\subsection{Getting The Code}
\label{section:build:get}

\gls{cppql} can be cloned from GitHub. Note that since the repository contains submodules, you have to pass an additional parameter. See Listing~\ref{lst:build:clone}.

\lstinputlisting[caption={Cloning.}, label={lst:build:clone}, language=sh]{snippets/clone.cmd}

\subsection{Configuration}
\label{section:build:config}

There are several CMake configuration variables. Below an overview.

\code{CPPQL\_BIND\_ZERO\_BASED\_INDICES} can be set to \code{ON} or \code{OFF}. When enabled, the indices passed to the various \code{bind} methods this library provides as wrappers around the C functions become 0-based. Note that this of course does not apply to any of the C functions, should you still use those.

\code{BUILD\_TESTS} can be set to \code{ON} or \code{OFF}. When enabled, an application containing tests is created. See Section~\ref{section:build:tests} for more information.

\subsection{Windows}
\label{section:build:windows}

Assuming you just cloned the repository to a folder named source, you can configure and build using the commands shown in Listing~\ref{lst:build:windows}.

\lstinputlisting[caption={Building on Windows using MSVC and vcpkg.}, label={lst:build:windows}, language=sh]{snippets/build_windows.cmd}

\subsection{Linux}
\label{section:build:linux}

Assuming you just cloned the repository to a folder named source, you can configure and build using the commands shown in Listing~\ref{lst:build:linux}.

\lstinputlisting[caption={Building on Linux using GCC and Conan.}, label={lst:build:linux}, language=sh]{snippets/build_linux.cmd}

\subsection{Running Tests}
\label{section:build:tests}

When \code{BUILD\_TESTS} is enabled, an application called \code{cppql\_test} is created (as well as some other targets the application depends on). Building this application requires \code{CPPQL\_BIND\_ZERO\_BASED\_INDICES} to be enabled as well. In order to run all tests, just run the application without any parameters. If \gls{cppql} was built successfully, they should all run without issue. The tests are written using the BetterTest~\cite{bettertest} library.
